\documentclass{beamer}

\usetheme{boxes}

 
\usecolortheme[RGB={34,170,34}]{structure} 
\usepackage{amsmath}
\usepackage{amssymb}
\usepackage{graphics}
\usepackage{multicol}
\usepackage{color}
\usepackage[absolute,overlay]{textpos}

\usepackage{framed,color}
\definecolor{shadecolor}{rgb}{255,127,0}
\setbeamertemplate{itemize item}[circle]

\definecolor{verde}{RGB}{34,170,34}


\setbeamercolor{uppercolgreen}{fg=white,bg=verde!90}
\setbeamercolor{lowercolgreen}{fg=black,bg=verde!20}

\title[Python]{Programming with Python - Day 2}

\subtitle[]{EOAS Software Carpentry Workshop }
\date[Sep 2015]{September 25th, 2015}

\begin{document}

\section*{Functions}

\begin{frame}[plain] 
\titlepage
\end{frame}

%---------------- FUNCTIONS LESSON STARTS HERE --------------------%

\subsection*{Creating Functions}

\begin{frame}[fragile]
\frametitle{Creating Functions - Defining a Function}
\begin{block}{Learning Goals}
\begin{enumerate}
\item Explain why we should divide programs into small, single-purpose functions.
\item Define a function that takes parameters.
\item Return a value from a function.
\end{enumerate}
\end{block}

\begin{block}{Example Code}
\begin{itemize}
\item
\begin{verbatim}
def fahr_to_kelvin(temp):
     return ((temp - 32) * (5/9)) + 273.15
\end{verbatim}
\item
\begin{verbatim}
def kelvin_to_celsius(temp):
    return temp - 273.15
\end{verbatim}

\item
\begin{verbatim}
def fahr_to_celsius(temp):
    temp_k = fahr_to_kelvin(temp)
    result = kelvin_to_celsius(temp_k)
    return result
\end{verbatim}
\end{itemize}
\end{block}
\end{frame}


\begin{frame}
\frametitle{Exercise}
Write a function called analyze that takes a filename as a parameter and displays the three graphs produced in the previous lesson, i.e., analyze('inflammation-01.csv') should produce the graphs already shown, while analyze('inflammation-02.csv') should produce corresponding graphs for the second data set. Hint: a function can just ``do'' something.  It doesn't necessarily need to return anything.
\end{frame}

\begin{frame}[fragile]
\frametitle{Solution}
\small{
\begin{verbatim}
def analyze(filename):
    data = np.loadtxt(fname=filename, delimiter=',')
    fig = plt.figure(figsize=(10.0, 3.0))

    axes1 = fig.add_subplot(1, 3, 1)
    axes2 = fig.add_subplot(1, 3, 2)
    axes3 = fig.add_subplot(1, 3, 3)

    axes1.set_ylabel('average')
    axes1.plot(data.mean(axis=0))

    axes2.set_ylabel('max')
    axes2.plot(data.max(axis=0))

    axes3.set_ylabel('min')
    axes3.plot(data.min(axis=0))

    fig.tight_layout()
    plt.show(fig)
\end{verbatim}}
\end{frame}


\begin{frame}[fragile]
\frametitle{Defining a Function}
\begin{verbatim}
def detect_problems(filename):

    data = np.loadtxt(fname=filename, delimiter=',')

    if data.max(axis=0)[0] == 0 and data.max(axis=0)[20] == 20:
        print('Suspicious looking maxima!')
    elif data.min(axis=0).sum() == 0:
        print('Minima add up to zero!')
    else:
        print('Seems OK!')
\end{verbatim}
\end{frame}


\begin{frame}[fragile]
\frametitle{Testing and Documentation}
\begin{block}{Learning Goal}
\begin{enumerate}
\setcounter{enumi}{2}
\item    Test and debug a function.
\end{enumerate}
\end{block}
\begin{block}{Example Code}
\small{
\begin{itemize}
\item
\begin{verbatim}
def centre(data, desired):
    return (data - data.mean()) + desired
\end{verbatim}
\item 
\begin{verbatim}
z = numpy.zeros((2,2))
\end{verbatim}
\item 
\begin{verbatim}
print centre(z, 3)
\end{verbatim}
\item 
\begin{verbatim}
print data.std() - centred.std()
\end{verbatim}
\item 
\begin{verbatim}
def center(data, desired):
  '''Return a new array containing the original data 
     centered around the desired value.'''
  return (data - data.mean()) + desired
\end{verbatim}
\item 
\begin{verbatim}
help(centre)
\end{verbatim}
\end{itemize}}
\end{block}
\end{frame}





\begin{frame}[fragile]
\frametitle{Defining Defaults}
\begin{block}{Learning Goals}
\begin{enumerate}
\setcounter{enumi}{5}
\item Set default values for function parameters.
\end{enumerate}
\end{block}
\begin{block}{Example Code}
\begin{itemize}
\item 
\begin{verbatim}
def center(data, desired = 0):
\end{verbatim}
\item
\begin{verbatim}
def display(a=1, b=2, c=3):
    print 'a:', a, 'b:', b, 'c:', c
print 'no parameters:'
display()
print 'one parameter:'
display(55)
print 'two parameters:'
display(55, 66)
\end{verbatim}
\item help(numpy.loadtxt)
\end{itemize}
\end{block}
\end{frame}

\begin{frame}[fragile]
\frametitle{Exercise}

``Adding" two strings produces their concatenation: \texttt{'a' + 'b'} is \texttt{'ab'}. Write a function called fence that takes two parameters called original and wrapper and returns a new string that has the wrapper character at the beginning and end of the original. A call to your function should look like this:
\begin{verbatim}
print(fence('name', '*'))
*name*
\end{verbatim}
\end{frame}

\begin{frame}[fragile]
\frametitle{Exercise}

``Adding" two strings produces their concatenation: \texttt{'a' + 'b'} is \texttt{'ab'}. Write a function called fence that takes two parameters called original and wrapper and returns a new string that has the wrapper character at the beginning and end of the original. A call to your function should look like this:
\begin{verbatim}
print(fence('name', '*'))
*name*
\end{verbatim}

\begin{block}{Solution}
\begin{verbatim}
def fence(original, wrapper):
    ```Returns a string with charcter wrapper added to the 
       beginning and end of string original.```
       
       return wrapper + original + wrapper
\end{verbatim}
\end{block}
\end{frame}


%---------------- COMMAND LINE PROGRAMS LESSON STARTS HERE --------------------%

\begin{frame}{Command-line programs}

\begin{enumerate}
    \item{Create a Python module containing functions that can be imported into notebooks and other modules.}
    \item{Use the values of command-line arguments in a program.}
    \item{Read data from standard input in a program so that it can be used in a pipeline.}
\end{enumerate}
\end{frame}

%-------------------------- Challenge 01 ------------------------------------------%


\begin{frame}{Switching to shell commands}

 \textcolor{verde}{\texttt{\$}} in front of a command that tells you to run that command in the shell rather than the Python interpreter
%
%\begin{figure}[htbp]
 %  \centering
  % \includegraphics[width=0.45\textwidth]{../figures/axis.png} 
%\end{figure}

\end{frame}

%-------------------------- Challenge 01 ------------------------------------------%

\begin{frame}{ }
Write a command-line program that does addition and subtraction:
\vspace{0.5cm}

\begin{beamerboxesrounded}[upper=uppercolgreen,lower=lowercolgreen,shadow=false]{}
\texttt{\& python arith.py add 1 2 }
\end{beamerboxesrounded}

\begin{beamerboxesrounded}[upper=uppercolgreen,lower=lowercolgreen,shadow=false]{}
\texttt{3}
\end{beamerboxesrounded}

\begin{beamerboxesrounded}[upper=uppercolgreen,lower=lowercolgreen,shadow=false]{}
\texttt{\& python arith.py subtract 3 4 }
\end{beamerboxesrounded}

\begin{beamerboxesrounded}[upper=uppercolgreen,lower=lowercolgreen,shadow=false]{}
\texttt{-1}
\end{beamerboxesrounded}


\end{frame}
%-------------------------- Challenge 02 ------------------------------------------%

\begin{frame}{ }
Rewrite \texttt{readings.py} so that it uses \texttt{-n, -m}, and \texttt{-x} instead of \texttt{--min, --mean}, and \texttt{--max} respectively. Is the code easier to read? Is the program easier to understand?

\end{frame}

%-------------------------- Challenge 03 ------------------------------------------%

\begin{frame}{ }
Separately, modify \texttt{readings.py} so that if no action is given it displays the means of the data.
\end{frame}

%-------------------------- the document ends here ----------------------------------%


\end{document}
