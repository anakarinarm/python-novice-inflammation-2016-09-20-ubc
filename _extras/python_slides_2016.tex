% sample.tex
\documentclass{beamer}

\usetheme{boxes}

\usecolortheme[RGB={34,170,34}]{structure}
\usepackage{amsmath}
\usepackage{amssymb}
\usepackage{graphics}
\usepackage{multicol}
\usepackage{color}
\usepackage[absolute,overlay]{textpos}
\usepackage{fancybox}

\usepackage{framed,color}
\definecolor{shadecolor}{rgb}{255,127,0}
\setbeamertemplate{itemize item}[circle]

\definecolor{verde}{RGB}{34,170,34}


\setbeamercolor{uppercolgreen}{fg=white,bg=verde!90}
\setbeamercolor{lowercolgreen}{fg=black,bg=verde!20}


%%%%%%%%%%%%%%%%%%%%%%%%%%%%%%%%%%%


\title[Python Lesson]{Programming with Python}

\subtitle[]{EOAS Software Carpentry Workshop }
\date[Sep 2016]{September 21st, 2016}


%------------------ the document starts here -------------------------%

\begin{document}
\bibliographystyle{plainnat}
\bibliography{bib/biblio}


%------------------ the titlepage frame-------- -------------------------%


\begin{frame}[plain]

\titlepage


\end{frame}


%---------------- the presentation begins here --------------------%

%-------------------------- Cool xkcd cartoon ------------------------------------------%

\begin{frame}

\begin{figure}[htbp]
   \centering
  \includegraphics[width=0.6\textwidth]{figs_slides/python_xkcd.png}
\end{figure}

\begin{textblock*}{7cm}(6.0cm,9.0cm)
		\centering
			\tiny{https://xkcd.com/353 }
\end{textblock*}



\end{frame}
%-------------------------- Intro ------------------------------------------%


\begin{frame}{Getting started}

\small{For our Python introduction we're going to pretend to be a researcher studying inflammation in patients who have been given a new treatment for arthritis.}
\vspace{0.5cm}

You need to download some files to follow this lesson:
\begin{enumerate}
 \item{Make a new folder in your Desktop called \texttt{python-novice-inflammation}.}
    \item{Download python-novice-inflammation-data.zip and move the file to this folder.}
    \item{If it's not unzipped yet, double-click on it to unzip it. You should end up with a new folder called data.}
   \item{You can access this folder from the Unix shell with:}
\end{enumerate}
\texttt{\$ cd \&\& cd Desktop/python-novice-inflammation/data}




\end{frame}

%-------------------------- jupyter notebook ------------------------------------------%

\begin{frame}{Launching Jupyter Notebook}

\small{There are several ways that we can use Python.  We're going to start with
a tool called Jupyter Notebook that runs in the browser.
In a shell window enter these commands:}

\vspace{0.5cm}

\begin{beamerboxesrounded}[upper=uppercolgreen,lower=lowercolgreen,shadow=false]{}
\texttt{\$ cd \\
\$ cd Desktop/python-novice-inflammation/data \\
\$ jupyter notebook}
\end{beamerboxesrounded}
\vspace{0.5cm}

\small{The shell window is now running a local web server for you.  Don't close it. You will need to open another shell window to do other command line things. Your browser should open to an "Jupyter:  Notebook" page showing a list of directories.}
\end{frame}

%-------------------------- AXIS ------------------------------------------%

\begin{frame}{Analyzing patient data}

\begin{enumerate}
    \item{Explain what a library is, and what libraries are used for.}
    \item{Load a Python library and use the things it contains.}
    \item{Read tabular data from a file into a program.}
    \item{Assign values to variables.}
    \item{Select individual values and subsections from data.}
   \end{enumerate}

\begin{multicols}{2}
\begin{itemize}
\item import numpy
\item numpy.loadtxt(fname=  delimiter=)
\item weight\_kg = 55
\item print('weight in kg:', weight\_kg)
\item weight\_lb = 2.2 * weight\_kg
\item type(data)
\item data.shape
\item data[0,0], data[0:1,0:1]
\item data[0:10:2,1]
\item data[:3,36:]
\end{itemize}
\end{multicols}
\end{frame}

%-------------------- intro 2 ---------------------------
\begin{frame}
\frametitle{Analyzing Patient Data cont'd}
\begin{enumerate}
\setcounter{enumi}{5}
\item    Perform operations on arrays of data.
\item    Display simple graphs.
\end{enumerate}
\begin{multicols}{2}
\begin{itemize}
\item data.mean()
\item data.std()
\item data.mean(axis=0)
\item \%matplotlib inline
\item from matplotlib import pyplot
\item pyplot.imshow(data)
\item pyplot.show()
\item pyplot.plot(ave\_inflammation)
\item import matplotlib import pyplot as plt
\item plt.subplot(1,3,1)
\item plt.ylabel('average')
\item plt.show()
\end{itemize}
\end{multicols}
\end{frame}




%-------------------------- AXIS ------------------------------------------%

\begin{frame}{Operations across an axis}

\begin{figure}[htbp]
   \centering
  \includegraphics[width=0.7\textwidth]{figs_slides/python-operations-across-axes.png}
\end{figure}

\end{frame}


%%-------------------------- Challenge 01 ------------------------------------------%

\begin{frame}{Exercise}
Create a single plot showing 1) the mean for each day and 2) the mean + 1 standard deviation for each day and 3) the mean - 1 standard deviation for each day.

\end{frame}

%% ---------------------------------- LOOPS ----------------------------------------%
\begin{frame}{Repeating actions with loops}

\begin{enumerate}
    \item{Explain what a for loop does.}
    \item Correctly write for loops to repeat simple calculations.
    \item{Trace changes to a loop variable as the loop runs.}
    \item{Trace changes to other variables as they are updated by a for loop.}
   \end{enumerate}

\begin{multicols}{2}
\begin{itemize}
\item for char in word:
\item len('aeiou')
\end{itemize}
\end{multicols}
\end{frame}
%-------------------------- Challenge 02 - loops ------------------------------------------%

\begin{frame}{ }

Python has a built-in function called range that creates a list of numbers: range(3) produces [0, 1, 2], range(2, 5) produces [2, 3, 4], and range(2, 10, 3) produces [2, 5, 8]. Using range, write a loop that prints the first three natural numbers:

\vspace{0.5cm}

\begin{beamerboxesrounded}[upper=uppercolgreen,lower=lowercolgreen,shadow=false]{}
\texttt{1\\
2\\
3}
\end{beamerboxesrounded}

\end{frame}

%-------------------------- Challenge 05 - loops - My Sol------------------------------------------%

\begin{frame}{ }

Python has a built-in function called range that creates a list of numbers: range(3) produces [0, 1, 2], range(2, 5) produces [2, 3, 4], and range(2, 10, 3) produces [2, 5, 8]. Using range, write a loop that prints the first three natural numbers:

\vspace{0.5cm}

\alert{One solution:}

\texttt{for num in range(1,4,1):}

\texttt{      print(num)}


\end{frame}


%-------------------------- Challenge 06 - loops ------------------------------------------%

\begin{frame}{ }

Exponentiation is built into Python:

\vspace{0.5cm}

\begin{beamerboxesrounded}[upper=uppercolgreen,lower=lowercolgreen,shadow=false]{}

\texttt{print(5**3)\\
125}
\end{beamerboxesrounded}

\vspace{0.5cm}

Write a loop that calculates the same result using multiplication (without exponentiation).
\end{frame}

%%-------------------------- Challenge 06 - loops - My Solution ------------------------------------------%

\begin{frame}{ }

Exponentiation is built into Python:

\vspace{0.5cm}

\begin{beamerboxesrounded}[upper=uppercolgreen,lower=lowercolgreen,shadow=false]{}

\texttt{print(5**3)\\
125}
\end{beamerboxesrounded}

\vspace{0.5cm}

Write a loop that calculates the same result using multiplication (without exponentiation)

\alert{One possible answer:}

\texttt{ans=1}\\
\texttt{for ii in range(1,4,1):}\\
\texttt{      ans=ans*5}\\
\texttt{print(ans)}

\end{frame}


\begin{frame}[label=lists]
  \frametitle{Storing Multiple Values in Lists}
  \begin{block}{Learning Goals}
    \begin{enumerate}
      \item Explain what a list is.
      \item Create and index lists of simple values.
    \end{enumerate}
  \end{block}
  \begin{block}{Lesson Commands}
    \begin{itemize}
      \item \texttt{odds = [1, 3, 5, 7]}
      \item \texttt{print(odds[0], odds[-1])}
      \item \texttt{for number in odds:}
      \item \texttt{names[1] = 'Darwin'}
      \item \texttt{odds.append(11)}
      \item \texttt{del odds[0]}
      \item \texttt{odds.reverse()}
    \end{itemize}
  \end{block}
\end{frame}


\begin{frame}
  \frametitle{Exercise}
  \begin{block}{Turn a String into a List}
    Use a for loop to convert the string \texttt{'hello'} into a list of letters:

    \texttt{['h', 'e', 'l', 'l', 'o']}

    Hint: You can create an empty list like this:

    \texttt{my\_list = []}
  \end{block}
\end{frame}

\againframe{lists}


\begin{frame}[label=glob]
  \frametitle{Analyzing Data from Multiple Files}
  \begin{block}{Learning Goals}
    \begin{enumerate}
      \item Use a library function to get a list of filenames that match a simple wildcard pattern.
      \item Use a for loop to process multiple files.
    \end{enumerate}
  \end{block}
  \begin{block}{Lesson Commands}
    \begin{itemize}
      \item \texttt{import glob}
      \item \texttt{filenames = glob.glob('*.csv')}
      \item \texttt{filenames[0:3]}
    \end{itemize}
  \end{block}
\end{frame}


\begin{frame}[label=conditionals]
  \frametitle{Making Choices}
  \begin{block}{Learning Goals}
    \begin{enumerate}
      \item Write conditional statements including `if`, `elif`, and `else` branches.
      \item Correctly evaluate expressions containing `and` and `or`.
    \end{enumerate}
  \end{block}
  \begin{block}{Lesson Commands}
    \begin{itemize}
      \item \texttt{if num > 100:}
      \item \texttt{else:}
      \item \texttt{if num > 0:}
      \item \texttt{elif num == 0:}
      \item \texttt{and}
      \item \texttt{or}
    \end{itemize}
  \end{block}
\end{frame}


\begin{frame}
  \frametitle{Python if/else Flowchart}
  \includegraphics[scale=3.75]{../fig/python-flowchart-conditional.png}
\end{frame}


\begin{frame}
  \frametitle{Exercise}
  \begin{block}{How Many Paths?}
    What will be printed if you run this code:

    \texttt{if 4 > 5:}\\
    \texttt{    print('A')}\\
    \texttt{elif 4 == 5:}\\
    \texttt{    print('B')}\\
    \texttt{elif 4 < 5:}\\
    \texttt{    print('C')}\\

    \begin{enumerate}
      \item A
      \item B
      \item C
      \item B and C
    \end{enumerate}

    Why did you pick your answer?

  \end{block}
\end{frame}


\begin{frame}
  \frametitle{Exercise}
  \begin{block}{Close Enough}
    Work with your partner to write some code that will print \texttt{True}
    if the value of variable \texttt{a} is within 10\% of the value of variable \texttt{b}
    and \texttt{False} otherwise.
    Test your code for positive values, negative values, and values that span zero.
  \end{block}
\end{frame}


\againframe{conditionals}

%---------------- FUNCTIONS LESSON STARTS HERE --------------------%

\subsection*{Creating Functions}

\begin{frame}[fragile]
\frametitle{Creating Functions - Defining a Function}
\begin{block}{Learning Goals}
\begin{enumerate}
\item Explain why we should divide programs into small, single-purpose functions.
\item Define a function that takes parameters.
\item Return a value from a function.
\end{enumerate}
\end{block}

\begin{block}{Example Code}
\begin{itemize}
\item
\begin{verbatim}
def fahr_to_kelvin(temp):
     return ((temp - 32) * (5/9)) + 273.15
\end{verbatim}
\item
\begin{verbatim}
def kelvin_to_celsius(temp):
    return temp - 273.15
\end{verbatim}

\item
\begin{verbatim}
def fahr_to_celsius(temp):
    temp_k = fahr_to_kelvin(temp)
    result = kelvin_to_celsius(temp_k)
    return result
\end{verbatim}
\end{itemize}
\end{block}
\end{frame}


\begin{frame}
\frametitle{Exercise}
Write a function called analyze that takes a filename as a parameter and displays the three graphs produced in the previous lesson, i.e., analyze('inflammation-01.csv') should produce the graphs already shown, while analyze('inflammation-02.csv') should produce corresponding graphs for the second data set. Hint: a function can just ``do'' something.  It doesn't necessarily need to return anything.
\end{frame}

\begin{frame}[fragile]
\frametitle{Solution}
\small{
\begin{verbatim}
def analyze(filename):
    data = np.loadtxt(fname=filename, delimiter=',')
    fig = plt.figure(figsize=(10.0, 3.0))

    axes1 = fig.add_subplot(1, 3, 1)
    axes2 = fig.add_subplot(1, 3, 2)
    axes3 = fig.add_subplot(1, 3, 3)

    axes1.set_ylabel('average')
    axes1.plot(data.mean(axis=0))

    axes2.set_ylabel('max')
    axes2.plot(data.max(axis=0))

    axes3.set_ylabel('min')
    axes3.plot(data.min(axis=0))

    fig.tight_layout()
    plt.show(fig)
\end{verbatim}}
\end{frame}


\begin{frame}[fragile]
\frametitle{Defining a Function}
\begin{verbatim}
def detect_problems(filename):

    data = np.loadtxt(fname=filename, delimiter=',')

    if data.max(axis=0)[0] == 0 and data.max(axis=0)[20] == 20:
        print('Suspicious looking maxima!')
    elif data.min(axis=0).sum() == 0:
        print('Minima add up to zero!')
    else:
        print('Seems OK!')
\end{verbatim}
\end{frame}


\begin{frame}[fragile]
\frametitle{Testing and Documentation}
\begin{block}{Learning Goal}
\begin{enumerate}
\setcounter{enumi}{2}
\item    Test and debug a function.
\end{enumerate}
\end{block}
\begin{block}{Example Code}
\small{
\begin{itemize}
\item
\begin{verbatim}
def centre(data, desired):
    return (data - data.mean()) + desired
\end{verbatim}
\item
\begin{verbatim}
z = numpy.zeros((2,2))
\end{verbatim}
\item
\begin{verbatim}
print centre(z, 3)
\end{verbatim}
\item
\begin{verbatim}
print data.std() - centred.std()
\end{verbatim}
\item
\begin{verbatim}
def center(data, desired):
  '''Return a new array containing the original data
     centered around the desired value.'''
  return (data - data.mean()) + desired
\end{verbatim}
\item
\begin{verbatim}
help(centre)
\end{verbatim}
\end{itemize}}
\end{block}
\end{frame}





\begin{frame}[fragile]
\frametitle{Defining Defaults}
\begin{block}{Learning Goals}
\begin{enumerate}
\setcounter{enumi}{5}
\item Set default values for function parameters.
\end{enumerate}
\end{block}
\begin{block}{Example Code}
\begin{itemize}
\item
\begin{verbatim}
def center(data, desired = 0):
\end{verbatim}
\item
\begin{verbatim}
def display(a=1, b=2, c=3):
    print 'a:', a, 'b:', b, 'c:', c
print 'no parameters:'
display()
print 'one parameter:'
display(55)
print 'two parameters:'
display(55, 66)
\end{verbatim}
\item help(numpy.loadtxt)
\end{itemize}
\end{block}
\end{frame}

\begin{frame}[fragile]
\frametitle{Exercise}

``Adding" two strings produces their concatenation: \texttt{'a' + 'b'} is \texttt{'ab'}. Write a function called fence that takes two parameters called original and wrapper and returns a new string that has the wrapper character at the beginning and end of the original. A call to your function should look like this:
\begin{verbatim}
print(fence('name', '*'))
*name*
\end{verbatim}
\end{frame}

\begin{frame}[fragile]
\frametitle{Exercise}

``Adding" two strings produces their concatenation: \texttt{'a' + 'b'} is \texttt{'ab'}. Write a function called fence that takes two parameters called original and wrapper and returns a new string that has the wrapper character at the beginning and end of the original. A call to your function should look like this:
\begin{verbatim}
print(fence('name', '*'))
*name*
\end{verbatim}

\begin{block}{Solution}
\begin{verbatim}
def fence(original, wrapper):
    ```Returns a string with charcter wrapper added to the
       beginning and end of string original.```

       return wrapper + original + wrapper
\end{verbatim}
\end{block}
\end{frame}


%---------------- TRACEBACKS & EXCEPTIONS LESSON STARTS HERE-------------------%


\begin{frame}[label=tracebacks_exceptions]
  \frametitle{Tracebacks and Exceptions}
  \begin{block}{Learning Goals}
    \begin{enumerate}
      \item Read a traceback, and determine the following relevant pieces of information:
      \begin{itemize}
        \item The file, function, and line number on which the error occurred
        \item The type of the error
        \item The error message
      \end{itemize}
    \item Describe the types of situations in which the following errors occur:
      \begin{itemize}
        \item\texttt{SyntaxError} and \texttt{IndentationError}
        \item\texttt{NameError}
        \item\texttt{IndexError}
        \item\texttt{FileNotFoundError}
      \end{itemize}
    \end{enumerate}
  \end{block}
\end{frame}


\begin{frame}[fragile]
  \frametitle{Exercise}
  \begin{block}{}
    Does this code raise an exception?
    If so, what is the name of the exception?
    \begin{verbatim}
      for x in range(10, -10, -1):
          print('inverse of', x, 'is', 1/x)
    \end{verbatim}
    Can you modify the code so that it does what is intended,
    but avoids the exception?
  \end{block}
\end{frame}


\begin{frame}[fragile]
  \frametitle{Try/Except Blocks}
  \begin{block}{Learning Goals}
    \begin{enumerate}
      \item Write error handling Python code using \texttt{try} and \texttt{except} statements.
    \end{enumerate}
  \end{block}
  \begin{block}{Lesson Commands}
    \begin{verbatim}
      try:
          # something that might go wrong
      except SomeError:
          # handle the error
    \end{verbatim}
  \end{block}
\end{frame}


%---------------- COMMAND LINE PROGRAMS LESSON STARTS HERE --------------------%

\begin{frame}{Command-line programs}
\begin{block}{Learning goals}
\begin{enumerate}
       \item{Use the values of command-line arguments in a program.}
    \item{Handle flags and files separately in a command-line program.}
    \item{Read data from standard input in a program so that it can be used in a pipeline.}
 \end{enumerate}
 \end{block}
 \begin{block}{Commands and functions}
  \texttt{sys.version \\
  sys.argv\\
  sys.stdin}
 \end{block}
\end{frame}

%--------------------------  ------------------------------------------%


\begin{frame}{Switching to shell commands}

 \textcolor{verde}{\texttt{\$}} in front of a command that tells you to run that command in the shell rather than the Python interpreter
%
%\begin{figure}[htbp]
 %  \centering
  % \includegraphics[width=0.45\textwidth]{../figures/axis.png}
%\end{figure}

\end{frame}

%-------- Challenge 01 ------------------------------------------%

\begin{frame}{ }
\begin{itemize}
\item Rewrite \texttt{readings.py} so that it uses \texttt{-n, -m}, and \texttt{-x} instead of \texttt{--min, --mean}, and \texttt{--max} respectively. Is the code easier to read? Is the program easier to understand?


\item Separately, modify \texttt{readings.py} so that if no action is given it displays the means of the data.
\end{itemize}
\end{frame}



%-------------------------- the document ends here ----------------------------------%

\end{document}
