% sample.tex
\documentclass{beamer}

\usetheme{boxes}

\usecolortheme[RGB={34,170,34}]{structure} 
\usepackage{amsmath}
\usepackage{amssymb}
\usepackage{graphics}
\usepackage{multicol}
\usepackage{fancybox}

\usepackage{framed,color}
\definecolor{shadecolor}{rgb}{255,127,0}

\definecolor{verde}{RGB}{34,170,34}


\setbeamercolor{uppercolgreen}{fg=white,bg=verde!90}
\setbeamercolor{lowercolgreen}{fg=black,bg=verde!20}
%%%%%%%%%%%%%%%%%%%%%%%%%%%%%%%%%%%


\title[Command line Python]{Command-line programs with Python}

\subtitle[]{EOAS Software Carpentry Workshop }
\date[Sep 2015]{September 25nd, 2015}


%------------------ the document starts here -------------------------%

\begin{document}
\bibliographystyle{plainnat}
\bibliography{bib/biblio}


%------------------ the titlepage frame-------- -------------------------%

 
\begin{frame}[plain]
  
\titlepage


\end{frame}


%---------------- the presentation begins here --------------------%

\begin{frame}{Command-line programs}

\begin{enumerate}
    \item{Create a Python module containing functions that can be imported into notebooks and other modules.}
    \item{Use the values of command-line arguments in a program.}
    \item{Read data from standard input in a program so that it can be used in a pipeline.}
\end{enumerate}
\end{frame}

%-------------------------- Challenge 01 ------------------------------------------%


\begin{frame}{Switching to shell commands}

 \textcolor{verde}{\texttt{\$}} in front of a command that tells you to run that command in the shell rather than the Python interpreter
%
%\begin{figure}[htbp]
 %  \centering
  % \includegraphics[width=0.45\textwidth]{../figures/axis.png} 
%\end{figure}

\end{frame}

%-------------------------- Challenge 01 ------------------------------------------%

\begin{frame}{ }
Write a command-line program that does addition and subtraction:
\vspace{0.5cm}

\begin{beamerboxesrounded}[upper=uppercolgreen,lower=lowercolgreen,shadow=false]{}
\texttt{\& python arith.py add 1 2 }
\end{beamerboxesrounded}

\begin{beamerboxesrounded}[upper=uppercolgreen,lower=lowercolgreen,shadow=false]{}
\texttt{3}
\end{beamerboxesrounded}

\begin{beamerboxesrounded}[upper=uppercolgreen,lower=lowercolgreen,shadow=false]{}
\texttt{\& python arith.py subtract 3 4 }
\end{beamerboxesrounded}

\begin{beamerboxesrounded}[upper=uppercolgreen,lower=lowercolgreen,shadow=false]{}
\texttt{-1}
\end{beamerboxesrounded}


\end{frame}
%-------------------------- Challenge 02 ------------------------------------------%

\begin{frame}{ }
Rewrite \texttt{readings.py} so that it uses \texttt{-n, -m}, and \texttt{-x} instead of \texttt{--min, --mean}, and \texttt{--max} respectively. Is the code easier to read? Is the program easier to understand?

\end{frame}

%-------------------------- Challenge 03 ------------------------------------------%

\begin{frame}{ }
Separately, modify \texttt{readings.py} so that if no action is given it displays the means of the data.
\end{frame}

%-------------------------- the document ends here ----------------------------------%

\end{document}
